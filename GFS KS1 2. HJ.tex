\documentclass{beamer}
\usetheme{Malmoe}
\usecolortheme{dove}
%\setbeamertemplate{navigation symbols}{}  %remove navigation symbols
%Deutsch
\usepackage[ngerman]{babel}
\usepackage[utf8]{inputenc}
%Mathe
\usepackage{amsmath}
%Einheiten
\usepackage{siunitx}
%Baumstrukturen
\usepackage{tikz}
\usepackage{tikz-qtree}
\usepackage{textcomp}   % allows \textrightarrow
%Bildquellen
\usepackage{varwidth}
\usepackage{graphicx}
\usepackage{hyperref}

\author{Florian Seligmann}
\title{Die Baupolitik des Augustus}
\date{\textbf{11.6.2018}}

\tikzset{
  invisible/.style={opacity=0},
  visible on/.style={alt={#1{}{invisible}}},
  alt/.code args={<#1>#2#3}{%
    \alt<#1>{\pgfkeysalso{#2}}{\pgfkeysalso{#3}}
  }
}

%Item mit Pfeil
\newcommand{\aitem}{%
\item[$\rightarrow$]
}

%Bildquellen
\newcommand*{\quelle}{%
  \tiny Quelle:
}

%Bilder
\newcommand{\bild}[3]{%
	\begin{figure}
	\centering
  \begin{varwidth}{\linewidth}
    \raggedleft
    \includegraphics[scale=#2]{#1}\\
    \quelle{\url{#3}}
  \end{varwidth}
\end{figure}
}

%Einheiten
\sisetup{
  locale = DE ,
  per-mode = symbol
}

%=====================================================================================================================================
\begin{document}

\frame{\titlepage}
\frame{\frametitle{Gliederung}
\tableofcontents
} %End of frame
%-------------------------------------------------------------------------------------------------------------------------------------


\section{Entwicklung der Baupolitik}

\subsection{Zeit der Bürgerkriege}

\frame{\frametitle{Bürgerkriege}
	\begin{itemize}
		\item 44 v. Chr. - 31 v. Chr.
		\pause
		\item Antonius von Ägypten aus
		\aitem Betonung Roms
		\pause
		\aitem Repräsentative Bauten in Rom
		\aitem Caesarkult
	\end{itemize}
}%eof

\frame{\frametitle{Beispiele zur Zeit der Bürgerkriege}
	%Mausoleum Augustus auf Marsfeld
	\bild{images/mausoleum.png}{0.35}{http://www.roma-antiqua.de/abbildungen/antikes_rom/marsfeld/2000_351.jpg}
}%eof

\frame{\frametitle{Beispiele zur Zeit der Bürgerkriege}
	%Mausoleum Augustus auf Marsfeld
	\bild{images/mausoleum_oben.png}{1.0}{https://www.realmofhistory.com/wp-content/uploads/2017/05/restoration-mausoleum-of-augustus-rome_3-770x437.jpg}
}%eof

\frame{\frametitle{Bürgerkriege}
	\begin{itemize}
		\item Bezug zu Apollo
		\pause
		\item Betonung der Ahnenreihe
		\pause
		\item Über Venus, Aeneas, Iulus Ascanius und Caesar
		\aitem Curia Iulia, Tempel des Divus Iulius auf dem Forum Romanum %Curia Iulia: Senatsgebäude, von Caesar begonnen; Divus Iulius: Tempel für vergöttlichten Caesar -> Lage am Forum Romanum
		\pause
		\item Bezug zu Romulus \textbf{$\rightarrow$} \glqq{}Neubgründer\grqq{}
		\aitem Wohnhaus auf dem Palatin bei der Hütte des Romulus
	\end{itemize}
}%eof

\subsection{Zeit des Prinzipats}

\frame{\frametitle{Prinzipat}
	\begin{itemize}
		\pause
		\item Fertigstellung vieler Bauwerke
		\item Erneuerung der Infrastruktur %-> Agrippa; "Cloaca Maxima"; Aquaeukte; Straßen; Brandschutz
		\item Erneuerung der Tempel %-> Rückkehr zu den Göttern
		\pause
		\item Siegesdenkmäler
	\end{itemize}
}%eof

\section{Ara Pacis}

\frame{\frametitle{Ara Pacis}
	\bild{images/ap_totale.png}{0.35}{https://www.bluffton.edu/homepages/facstaff/sullivanm/italy/rome/arapacis/0081.jpg}
}%eof


\frame{\frametitle{Ara Pacis Augustae - Eckdaten}
	\begin{itemize}
		\item 13 v. Chr. vom Senat in Auftrag gegeben
		\item Geweiht am 30. Januar 9 v. Chr. (Geburtstag der Livia)
		\pause
		\item 11.63m x 10.62m
	\end{itemize}
} %eof

\frame{\frametitle{Fundgeschichte}
	\begin{itemize}
		\item Unter Tiberschlamm begraben
		\pause
		\item 1568 Erste Entdeckungen
		\item 1879 Indentifikation als Ara Pacis
		\pause
		\item Ausgrabungen 1903 \& 1937/38 %Mussolini -> Identifikation mit Augustus
		\item Heute Rekonstruktion
	\end{itemize}
}%eof

\frame{\frametitle{Aufgaben}
	\begin{itemize}
		\item Rückblick auf Griechenland \& die Republik
		\item Versprechen eines \glqq{}Goldenen Zeitalters\grqq{} 
	\end{itemize}
}%eof

\frame{\frametitle{Gestaltung}
	\begin{itemize}
		\item Interpretation sehr umstritten
		\item Einteilung %Oben Szenen, Handlungen; unten Natur, Wildheit, Roheit
	\end{itemize}
}%eof

\end{document}